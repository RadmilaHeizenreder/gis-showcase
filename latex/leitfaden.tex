\section*{Leitfaden}
Abstract:
•	Das Abstract fasst die wesentlichen Inhalte Ihrer Bachelorarbeit zusammen, ohne diese zu werten oder zu interpretieren.
•	Bietet einen groben Überblick über Ihre Fragestellung, Ihr Vorgehen und Ihre zentralen Ergebnisse
•	Wird am Ende geschrieben.

Das Abstract enthält:
1.	-Hintergrundinformationen (z. B. Ausgangslage, Fragestellung und Zielsetzung der Arbeit, Relevanz/
2.	Forschungskontext, entspricht der Einleitung der Arbeit)
3.	-Forschungsmethoden/ Vorgehen, evtl. kurze Benennung der Thesen, bei empirischen Studien auch Angaben zu den Daten wie etwa die Charakteristika der Stichprobe.
4.	-Ergebnisse
5.	-Schlussfolgerungen, Implikationen oder Anwendungsmöglichkeiten (entspricht dem Diskussionsteil der Arbeit) 
6.	-Inhalt knapp, vollständig und präzise wiedergeben
7.	-Fokus auf das Wesentliche
8.	-Objektiv, einfach, verständlich formulieren
9.	-Keine Nennung des Titels, keine Zitate
10.	-Umfang in der Regel 150 bis 250 Wörter ->ggf. Betreuer*in fragen

Insgesamt ca. 5 Sätze zu folgenden Punkten (keine bestimmte Reihenfolge): 
1.	Was war das Ziel und die Fragestellung der Arbeit? 
2.	Welche Methode(n) wurden verwendet? 
3.	Was waren die wichtigsten Konzepte und Quellen, auf denen die Arbeit basiert? 
4.	Was waren die wichtigsten Ergebnisse der Arbeit, und was folgt daraus? 
5.	Ggf. kurze Verortung des Themas in einem allgemeinen Kontext



Einleitung
•	Um welchen Themenbereich handelt es sich?
•	Warum ist es relevant, sich mit diesem Themenbereich zu befassen?
•	Was hat die Forschung zu diesem Themenbereich an Erkenntnissen gewonnen?
•	Wie wurden die vorhandenen Erkenntnisse gewonnen?
•	Welche Fragen sind bislang offen geblieben?
•	Welche dieser offenen Fragen ist Gegenstand der vorliegenden Untersuchung?
•	Wie werden die neuen Erkenntnisse gewonnen?
•	Welche neuen Erkenntnisse sollen gewonnen werden?
•	Wie sind die zu erwartenden neuen Erkenntnisse im Zusammenhang mit den bereits vorhandenen Erkenntnissen einzuschaetzen?
•	Welche Rahmenbedingungen muessen beachtet werden?
•	Wie ist die Arbeit angelegt?


	Verbindung zum Allgemeinen herstellen - In welchem übergeordneten Thema und vor welcher fachspezifischen Herausforderung ist die Arbeit verortet? Hier sollten Sie ein wissenschaftliches, aber u.U. fachfremdes Lesepublikum abholen und in Ihren Bereich des Themas einführen.
	Forschungsstand zusammenfassen und Forschungslücke identifizieren - Was wurde zu diesem Themenbereich schon untersucht, und was noch nicht? Hier also eine kurze Zusammenfassung des Kapitels Forschungsstand und theoretische Grundlagen. 
	Ziel und Fragestellung benennen - Welchen Teilaspekt des Themas möchten Sie innerhalb dieser Arbeit untersuchen und warum? Was unterscheidet Ihre Arbeit von anderen Arbeiten zu ähnlichen Themen? 
	Ggf. Hypothese formulieren - Formulieren Sie noch vor der Durchführung Deiner Untersuchung eine begründete Vermutung, welche Antwort auf Ihre Fragestellung anhand des Forschungsstandes am wahrscheinlichsten ist. (Wenn Sie eine explorative Arbeit schreiben, gehen Sie anders vor: dann ist es Ziel der Arbeit, eine Hypothese anhand der Untersuchungsergebnisse zu formulieren.) 
	Überblick zum Aufbau geben - Kurze Info über Kapitelinhalte: Was sollen sie zur Beantwortung der Forschungsfrage beitragen?
	Abkürzungen erklären - Wenn Sie Abkürzungen verwenden, schreiben Sie sie das erste Mal ganz aus. Gemeint sind damit nicht reguläre Abkürzungen wie ‚z.B.‘ für ‚zum Beispiel’, sondern Akronyme wie ‚AMA‘ für ‚American Marketing Association’. Bei vielen Abkürzungen sollten Sie ein separates Abkürzungsverzeichnis anlegen.


\begin{abstract}
Jedes Kapitel sollte eine Überleitung und wenn möglich auch ein Zwischenfazit oder eine kurze Zusammenfassung haben: 
\begin{itemize}
    \item Welchen logischen Bezug hat das aktuelle Kapitel zum vorherigen oder nächsten Kapitel?
    \item Was waren die wichtigsten Erkenntnisse und die Hauptaussage in diesem Kapitel?
    \item Was trägt dieses Kapitel zur Beantwortung der Forschungsfrage bei?
\end{itemize}
\end{abstract}


Code-Snippsel:
Dies ist ein Beispiel für \Verb|code in einer Zeile|, der auch Sonderzeichen wie

zitieren \citep{Bleek.2005} ==> (Bleek 2005)==================
zitieren \citet{Bleek.2005} ==> Bleek (2005)==================
zitieren \citep[S.~373]{Lange.2020} ==================


(Abs.~\ref{sec:Form.io})