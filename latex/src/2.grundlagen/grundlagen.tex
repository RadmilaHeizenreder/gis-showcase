\chapter{Grundlagen}

Das Hauptziel dieses Kapitels besteht darin, eine solide Grundlage für das spätere Implementieren des Showcases zu schaffen, indem sowohl die theoretischen als auch die technologischen Grundlagen vertieft betrachtet werden. Zuerst wird eine eingehende Untersuchung digitaler Formulare, ihrer Struktur und ihrer Bestandteile durchgeführt. Anschließend wird das Formularmanagementsystem form.io vorgestellt, seine Technologie und Systemkomponenten analysiert und mit anderen Systemen verglichen. Darüber hinaus liegt ein besonderer Schwerpunkt auf der ausführlichen Recherche zu Geodaten und Geoinformationssystemen, um ein umfassendes Verständnis für deren Struktur, Komponenten und Funktionsweise zu entwickeln. Ein weiterer wichtiger Aspekt ist die Forschung und Auswahl einer geeigneten Datenbank zur Speicherung und Verarbeitung von Geodaten.



% Am besten ist es, wenn der Routingservice (router-ors) ein Standardformat für die Route rausgibt, hierfür eignet sich GeoJSON am besten. Wenn wir später doch mal einen anderen Routingservice anbinden - bspw. Google Maps API oder "von Hand zeichnen", wird der Service vielleicht anders sein, ein GeoJSON wird aber immer möglich sein.
% Das ist ein allgemein bekanntes Austauschformat und kann meines Wissens sogar direkt in das Linestring-Feld der Datenbank eingelesen werden.

% \paragraph{\href{https://geoportal-hamburg.de/schulinfosystem/?isinitopen=filter}{Beispiel-Hamburg}}


