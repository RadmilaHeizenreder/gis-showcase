\chapter{Grundlagen}
\label{sec:Grundlagen}
Das Hauptziel dieses Kapitels besteht darin, die theoretischen und begrifflichen Grundlagen für das Implementieren des Showcases zu schaffen. Es werden Formulare und Formularmanagementsystem Form.io, Geoinformationssystem, Geodaten und -banken näher angeschaut. Abschließen wird Openlayers-Bibliothek vorgestellt.



% Am besten ist es, wenn der Routingservice (router-ors) ein Standardformat für die Route rausgibt, hierfür eignet sich GeoJSON am besten. Wenn wir später doch mal einen anderen Routingservice anbinden - bspw. Google Maps API oder "von Hand zeichnen", wird der Service vielleicht anders sein, ein GeoJSON wird aber immer möglich sein.
% Das ist ein allgemein bekanntes Austauschformat und kann meines Wissens sogar direkt in das Linestring-Feld der Datenbank eingelesen werden.

% \paragraph{\href{https://geoportal-hamburg.de/schulinfosystem/?isinitopen=filter}{Beispiel-Hamburg}}


