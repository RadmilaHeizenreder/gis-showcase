\section{Vorgehensweise und Struktur}
%Recherche - Anforderungen - Technologien
Im dem theoretischen Teil der Arbeit werden die Grundlagen für Formulare der öffentlichen Verwaltung, FMS und deren Inhalte beschrieben. Es wird auf die Technologie, Architektur und Systemkomponenten des FMS Form.io eingegangen. Darauffolgend werden Geodaten und deren Quellen, gängige Geoinformationssysteme und deren Verwendung vorgestellt. In diesem Abschnitt werden die Systeme verglichen und Entscheidungen für die Auswahl der Technologien zur Durchführung der Konzeptentwicklung des Showcases getroffen.

Im Hauptteil des Kapitel 3 werden die Anforderungen an der Showcase auf Basis der erarbeiteten Grundlagen und den vom Auftraggeber vorgeschriebenen funktionalen Anforderungen an den Prototyp aufgelistet und um die nicht-funktionalen Anforderungen ergänzt. Darauf aufbauend wird im Kapitel 4 die Auswahl der Komponenten begründet und das Lösungskonzept sowie seine Architektur erarbeitet.

In Kapitel 5 wird die Implementierung des Showcases erläutert, indem auf die Struktur und die wesentlichen Details mit Hilfe des Quellcodes eingegangen wird. Die Arbeitergebnisse werden in der Kapitel 6 ausgewertet. Es sollen Erkenntnisse daraus gezogen werden, sowohl die Herausforderungen in der Umsetzung als auch den Mehrwert für den Auftraggeber bzw. den Nutzer zu betrachten, um so Handlungsempfehlungen für zukünftige Entwicklungen zu bieten. Abschließend ziehe ich in Kapitel 7 ein Fazit meiner Arbeit.


