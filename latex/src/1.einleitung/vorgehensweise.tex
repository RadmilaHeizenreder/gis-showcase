\section{Vorgehensweise und Struktur}
%Recherche - Anforderungen - Technologien
Im dem theoretischen Teil der Arbeit werden die Grundlagen für Formulare der öffentlichen Verwaltung, FMS und deren Inhalte beschrieben. Es werden auf die Technologie, Architektur und Systemkomponente des FMS Form.io eingegangen. Darauffolgend werden Geodaten und deren Quellen, gängige Geoinformationssysteme und deren Verwendung vorgestellt. In dem Abschnitt werden die Systeme verglichen und Entscheidungen für die Auswahl der Technologien zur Durchführung der Konzptentwicklung des Showcases getroffen.

Im Hauptteil der Kapitel 3 Anforderungen werden auf der Basis der erarbeiteten Grundlagen die vom Auftraggeber vorgeschriebene funktionale Anforderungen an den Prototyp aufgelistet und die nicht-funktionale Anforderungen aufgestellt.
Darauf wird im Kapitel 4 die Auswahl der Komponente, die auf der Anforderungen des Auftraggebers basieren, begründet und das Lösungskonzept sowie seine Architektur erarbeitet.

In der Kapitel 5 wird die Implementierung des Showcases erläutert, indem die Struktur und die wesentliche Details mit Hilfe des Quellcodes eingegangen wird. Die Arbeit wird in der Kapitel 6 ausgewertet. Die wichtigen Erkenntnisse werden daraus gezogen, die Herausforderungen und die Vorteile für Auftraggeber betrachtet und weitere Handlungsempfehlungen darbieten. Abschließend wird ein Ausblick auf weitere Anwendungsfelder gegeben.

Abschließend ziehe ich in Kapitel 7 ein Fazit meiner Arbeit.


