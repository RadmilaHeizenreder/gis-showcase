\section{Motivation}
%Verwaltungsdigitalisierung in Deutschland
%OZG und Telekom
%Ziel - Formulare eine weitere Funktionalität zu schaffen\\
In unserer Zeit, die nicht nur durch den digitalen Wandel, sondern auch durch die Herausforderungen der Covid-19-Pandemie geprägt ist, hat sich die Bedeutung digitaler Technologien dramatisch verstärkt. Mit der stetigen Weiterentwicklung mobiler Endgeräten hat sich unser Alltag tiefgreifend geändert. Wir nutzen diese Technologien nicht nur für Kommunikation und Unterhaltung, sondern auch für Bildung und Geschäftliches. Insbesondere während der Pandemie hat sich gezeigt, wie unverzichtbar digitale Lösungen für die Aufrechterhaltung des sozialen und wirtschaftlichen Lebens sind.\\


In diesem Kontext streben auch Behörde danach, ihre Dienste online anzubieten, um mit der globalen Entwicklungen und den durch die Pandemie verstärkten Anforderungen an digitale Zugänglichkeit Schritt zu halten. In Deutschland bildet das 2017 einfgeführte Onlinezugangsgesetz (OZG) die rechtliche Grundlage für diese Transformation. Das OZG, ein Gesetz zur Verbesserung des Onlinezugangs zu Verwaltungsleistungen, verpflichtet seitdem Bund, Länder und Kommunalbehörden dazu, ihre Dienstleistungen auch elektronisch über Verwaltungsportale bereitszustellen \citep{bundesministerium_des_innern_und_fur_heimat_onlinezugangsgesetz_2017}. Ziel des OZG ist es, den Bürgern und Unternehmen einen einfacheren und effizienteren Zugang zu Verwaltungsleistugen zu bieten. Es ermöglicht den Bürgern, Anträge und Formulare sowie andere Verwaltungsprozesse online zu erledigen.\\

Eines der wichtigesten Schlüsselelemente in dieser digitalen Transformation ist ein leistungsfähiges Formularmanagementsystem (FMS). Hier spielt das von der Telekom MMS GmbH weiterentwickelte Open Source form.io \citep{formio} eine zentrale Rolle für die Entwicklung des Verwaltungsportals in NRW. Es unterstützt die Digitalisierung von Papierformularen und trägt zu einer effizienten, modernen IT-Infrastruktur bei.\\

``Digitalisierung bedeutet in der Folge, analoge Informationen in digitale Daten umzuwandeln'' \cite[S.~90]{markus_auf_2022}
Die Digitalisierung der Papierformulare, welche durch das FMS form.io ermöglicht wird, hat das Potenzial, nicht nur die Effizienz zu steigern, sondern auch die Benutzererfahrung zu verbessern. Ein innovativer Ansatz in diesem Bereich ist die Integration dynamische Elemente in digitale Formulare. Beispielsweise können abhängig von den Benutzereingaben unterschiedliche Eingabebereiche ein- oder ausgeblendet werden. Weiterhin bietet das FMS die Möglichkeit, Daten aus Datenbanken abzurufen und die Eingabefelder vorzubefüllen. Die Antragsstellende können somit relevante Daten einsehen, bearbeiten und direkt im Formular übernehmen.\\



% wie digitalisieren sich behörde und ihre dienste? 



 