\section{Motivation}
\label{sec:motivation}
%Verwaltungsdigitalisierung in Deutschland
%OZG und Telekom
%Ziel - Formulare eine weitere Funktionalität zu schaffen\\

% Mich würde unter der Überschrift Motivation, aber schon interessieren, warum Du genau dieses Projekt machst. Dass es um ein GIS in form.io geht erwähnst Du hier noch gar nicht; Vorschlag: betone etwas stärker, dass es darum geht, dem Bürger die Leistungen nicht nur digital sondern auch einfach/komfortabel/übersichtlich anzubieten. Daher will man Komfortfunktionen wie Once-only-Authentifizeirung, automatische Datenübernahme etc. anbieten oder aber auch übersichtliche Karten – und das machst jetzt Du hier, da es noch kein andere hat…

In unserer Zeit, die nicht nur durch den digitalen Wandel, sondern auch durch die Herausforderungen der Covid-19-Pandemie geprägt ist, hat sich die Bedeutung digitaler Technologien dramatisch verstärkt. Die stetig weiterentwickelte mobile Endgeräte nutzen wir nicht nur für Kommunikation und Unterhaltung, sondern auch für Bildung und Geschäftliches. Insbesondere während der Pandemie hat sich gezeigt, wie unverzichtbar digitale Lösungen für die Aufrechterhaltung des sozialen und wirtschaftlichen Lebens sind.\\

In diesem Kontext streben auch Behörde danach, ihre Dienste online anzubieten, um mit den globalen Entwicklungen und den durch die Pandemie verstärkten Anforderungen an digitale Zugänglichkeit Schritt zu halten. In Deutschland bildet das 2017 einfgeführte Onlinezugangsgesetz (OZG) die rechtliche Grundlage für diese Transformation. Das OZG, ein Gesetz zur Verbesserung des Onlinezugangs zu Verwaltungsleistungen, verpflichtet seitdem Bund, Länder und Kommunalbehörden dazu, ihre Dienstleistungen auch elektronisch über Verwaltungsportale bereitszustellen \citep{bundesministerium_des_innern_und_fur_heimat_onlinezugangsgesetz_2017}. Ziel des OZG ist es, den BürgerInnen und Unternehmen einen einfacheren und effizienteren Zugang zu Verwaltungsleistungen zu bieten. Es ermöglicht, die Angebote der öffentlichen Verwaltungen online zu nutzen, indem sie ihre Anliegen mittels digitalisierter Antragsformulare an die Verwaltung richten. Eines der wichtigsten Schlüsselelemente in dieser digitalen Transformation ist ein leistungsfähiges Formularmanagementsystem (Abs. ~\ref{sec:Form.io}). Es unterstützt die Digitalisierung von Papierformularen und trägt zu einer effizienten, modernen IT-Infrastruktur bei.\\

``Digitalisierung bedeutet in der Folge, analoge Informationen in digitale Daten umzuwandeln'' \cite[S.~90]{markus_auf_2022}. Die Digitalisierung der Papierformulare, hat das Potenzial, nicht nur die Effizienz zu steigern, sondern auch die Benutzererfahrung zu verbessern. Ein innovativer Ansatz in diesem Bereich ist die Integration dynamische Elemente in digitale Formulare. Beispielsweise können abhängig von den Benutzereingaben unterschiedliche Eingabebereiche ein- oder ausgeblendet werden. Weiterhin bietet das FMS die Möglichkeit, Daten aus Datenbanken abzurufen und die Eingabefelder vorzubefüllen. Die Antragsstellende können somit relevante Daten einsehen, bearbeiten und direkt im Formular übernehmen.\\

Die Deutsche Telekom MMS GmbH, im Folgenden nur ``Telekom MMS'' genannt, leistet einen wesentlichen Beitrag zur Verwaltungsdigitalisierung \citep{mms_moderne_verwaltung_2023}. Dabei konzentriert sie sich darauf, die Interaktionen zwischen BürgerInnen und der Verwaltung zu optimieren und zu verbessern. Dies umfasst die Implementierung von Komfortfunktionen wie Verwaltungsportale, Online-Dienste, Once-only-Authentifizierung, automatischen Datenübernahmen und sicheren Zahlungslösungen. In diesem Kontext stellt die Integration raumbezogener Daten und interaktiver Karten in Online-Formulare eine innovative und bisher noch wenig genutzte Möglichkeit dar. Durch die Einbindung von Funktionen des Geoinformationssystems (Abs.~\ref{sec:GIS}) in Formulare können BürgerInnen direkt mit Karten interagieren, beispielsweise um Standorte zu markieren, Routen zu planen oder geografische Informationen abzurufen. Solche Funktionalität erweitert das Spektrum der digitalen Interaktionen erheblich und bietet BürgerInnen sowie die Verwaltung einen erheblichen Mehrwert.\\

Diese Vision der Integration von Geoinformationssystemen (GIS) in digitale Bürgerdienstformulare ist neuartig und bisher wenig erforscht. Sie bietet spannende Perspektiven für die Entwicklung digitaler Verwaltungsdienste. Diese fortschrittliche Funktionalität und ihr großes Potenzial sind die Hauptgründe, dieses Thema in dieser Arbeit intensiv zu behandeln.