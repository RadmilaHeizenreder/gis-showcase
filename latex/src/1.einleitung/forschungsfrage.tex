\section{Ausgangssituation und Aufgabenstellung}
\label{sec:forschung}
% ``Die Digitalisierung der Verwaltung basiert auf den Visionen, die bereits vor vielen Jahren u.a. in Bezug auf medienbruchfreie, ubiquitäre, digitale Antragsmöglichkeiten für Bürger:innen skizziert wurde. Die Digitalisierung folgt den Digitalisierungsstrategien der Beteiligten u.a. auf den föderalen Ebenen.'' \cite[S.~21]{lohmann_architekturrahmen_2021}\\

Seit einigen Jahren wird das Verwaltungsdenken von dem Schlagwort Kundenorientierung beflügelt. Gesucht sind Herangehensweisen und Verfahren, die bei der Entwicklung von E-Government-Dienstleistungen systematisch und nachhaltig die Anliegen der potenziellen Nutzer wie Bürger:innen oder Unternehmen im jeweiligen Zuständigkeitsbereich einbeziehen und zu gesicherter Akzeptanz der Online-Angebote führen \citep{bleek_steuerungsmodell_2005}.\\

Die Deutsche Telekom MMS GmbH, im Folgenden nur „Telekom MMS“ genannt, entwickelt seit einigen Jahren Verwaltungsportale für NRW. Einige Formulare dort sind mit FMS Form.io modelliert worden und es wurden einige Funktionalitäten realisiert, die das Ausfüllen des Antragskomfortabler für die BürgerInnen machen. Im Rahmen dieser Bachelorarbeit werde ich für das Unternehmen ein Showcase entwerfen und den Funktionsumfang des FMS erweitern, indem eine neue Funktionalität ins Formular integriert wird.\\

Heutzutage ist unsere Welt ohne Maps, Geodaten und Navigation nicht mehr vorstellbar. Die Nutzung von Geoinformationen spielt mittlerweile in allen Bereichen – sowohl in der Wissenschaft, der öffentlichen Verwaltung, der Wirtschaft, Politik als auch im privaten Leben – eine wichtige Rolle. Daher wird in der Arbeit die Forschungsfrage gestellt und untersucht, ob und welches geografische Informationssystem in das FMS form.io als neue Funktionalität integriert werden kann und inwieweit dieses die Verwaltungsprozesse unterstützt.\\

Ich schreibe somit über die Konzeption und Implementierung eines Showcases zur Integration eines GIS in das FMS Form.io am Beispielformular Schulanmeldung in der Stadt Münster. Ich möchte verstehen, welche technischen und praktischen Herausforderungen bei der Einbettung geografischer Informationssysteme in Online-Formulare bestehen und wie diese effizient gelöst werden können. Dabei möchte ich herausfinden, welche Vorteile die GIS-Integration für die Datenvisualisierung und -verarbeitung im FMS bietet und wie dies die Benutzerinteraktion durch den exemplarischen Anwendungsfall verbessern kann.