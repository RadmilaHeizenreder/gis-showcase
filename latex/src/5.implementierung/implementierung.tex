\chapter{Implementierung des GIS-Showcases}

als erstes habe ich angefangen, mir Geodaten von Schulen in Münster zu suchen.
Sobald ich diese gefunden
% \begin{enumerate}
%     \item Wie kann ich meine Frage beantworten?
%     \item Welche Methode in welchem Zeitraum mit wievielen Menschen / Quellen / ...
%     \item werde ich anwenden?
%     \item Wie setze ich die Anforderungen um?
% \end{enumerate}
% Implementierung des Showcase
% Bei der Entwicklung des Showcase, der die Integration eines GIS in ein form.io-Formular demonstriert, wurden folgende Schlüsselelemente von OpenLayers genutzt:

% Verschiedene Kartenquellen: Durch die Einbindung diverser Raster- und Vektordatenquellen konnte eine reichhaltige und informative Kartenbasis geschaffen werden. Die Nutzung von Quellen wie OpenStreetMap und GeoJSON ermöglichte eine detaillierte und genaue Darstellung der relevanten geografischen Informationen.

% Layer-Management: Durch das geschickte Management verschiedener Layer konnten dynamische und benutzerfreundliche Kartenansichten erstellt werden. Dies war besonders wichtig, um unterschiedliche Datensätze wie Schulstandorte, Verkehrsnetze und geografische Besonderheiten effektiv zu visualisieren.

% Interaktive Bedienelemente und Overlays: Die Implementierung von Zoom- und Drehfunktionen, zusammen mit benutzerdefinierten Overlays wie Tooltips und Markern, trug dazu bei, die Nutzerinteraktion zu verbessern und wichtige Informationen anschaulich zu präsentieren.

% Individuelle Stilisierung: Das maßgeschneiderte Styling ermöglichte eine klare Hervorhebung spezifischer Geoobjekte, was für die Nutzerführung und das Verständnis der Karteninhalte entscheidend war.

% Geographische Berechnungen: Funktionen wie Distanzmessung und Flächenberechnung waren essentiell, um zusätzliche Informationen und Kontext zu den dargestellten Daten zu liefern.

% Ereignisgesteuerte Interaktionen: Die Einbindung von Ereignis-Listenern ermöglichte eine reaktionsschnelle und interaktive Kartenanwendung, die auf Nutzereingaben in Echtzeit reagiert.

% Fazit der Implementierung
% Die erfolgreiche Implementierung dieser Funktionen in den Showcase zeigt, wie OpenLayers effektiv eingesetzt werden kann, um eine anspruchsvolle und benutzerorientierte GIS-Lösung in ein Online-Formular zu integrieren. Dieses Beispiel unterstreicht die Fähigkeit von OpenLayers, anspruchsvolle GIS-Anforderungen in webbasierten Anwendungen zu erfüllen und bietet gleichzeitig einen praktischen Rahmen für die Verwendung dieser Technologie in ähnlichen Projekten.



