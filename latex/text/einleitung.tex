\chapter{Einleitung}
\label{cha:einleitung}
\section{Motivation}
%Verwaltungsdigitalisierung in Deutschland
%Telekom und Online-Dienste der Behörde
%Behörde werden attraktiver\\

Neben Klimawandel verfolgt uns digitaler Wandel. Seitdem wir eine neue Generation der Mobilendgeräten haben, wird vieles über diese Geräte abgewickelt: telefonieren, Filme und TV schauen, Bücher und Zeitung lesen, unsere Finanzen verwalten, Verträge abschließen. Digital ist nicht mehr weg zu denken, so dass die öffentliche Verwaltung mithalten muss, um mit der Weltentwicklung Schritt zu halten. Nun seit einigen Jahren ist Deutschland ganz fleißig bei der Digitalisierung der Verwaltungsleistungen. Daher ist es sehr wichtig die online-dienste der Behörden digital anzubieten. z.b Führerschein  oder Baugenehmigung beantragen ohne vor Ort zu sein.

Dafür wurde Onlinezugangsgesetz (OZG) als Grundlage der Verwaltungsdigitalisierung verabschiedet. ``Das OZG ist die rechtliche Grundlage für das bis dato größte Modernisierungsprojekt der öffentlichen Verwaltung seit Bestehen der Bundesrepublik. Im OZG werden die Digitalisierung von Verwaltungsleistungen sowie deren Bereitstellung über Verwaltungsportale geregelt. Für die Umsetzung des Gesetzes bedarf es unter anderem einer effizienten Arbeitsteilung, einer modernen IT-Infrastruktur sowie gemeinsamer Standards zwischen Bund, Ländern und Kommunen.'' \cite{bundesministerium_des_innern_und_fur_heimat_onlinezugangsgesetz_2017}


\section{Ausgangssituation und Aufgabenstellung}
%hier beschreibe ich die forschungsfrage und setze das Ziel, wie ich die frage beantworten kann und welche Methode ich verwende
%Online Formulare brauchen mehr Funktionen
%KOnzeptionstrategie für einen Showcase eines Formulars
``Die Digitalisierung der Verwaltung basiert auf den Visionen, die bereits vor vielen Jahren u.a. in Bezug auf medienbruchfreie, ubiquitäre, digitale Antragsmöglichkeiten für Bürger:innen skizziert wurde. Die Digitalisierung folgt den Digitalisierungsstrategien der Beteiligten u.a. auf den föderalen Ebenen.''\cite[S.~21]{lohmann_architekturrahmen_2021}

Durch die Digitalisierung der öffentlichen Verwaltung wurde den Bürger:innen die Online-Dienste der Verwaltung zu Verfügung gestellt. Diese Dienste sind online Formulare. Man kann Formulare online ausfüllen und an die Behörde abschicken ohne vor Ohr zu sein. 
online Formulare spielen eine wichtige Rolle bei der Verwaltungsdigitalisierung. die sollen bequem, verständlich und schnell sein. die sollte bei der Ausfüllen validierbar sein wie zum beispiel die Adressen könnten auf die aktualität überprüft werden. Die Daten können vorbefüllbar und vordefiniert sein. Was wenn es um solche Daten geht, die spezifisch sind wie z.b. Geodaten. Es gibt zahlreiche Programme, die es ermöglichen die Geodaten zu visualisieren, wie z.b Geoinformationsysteme. Die sind sehr komplex und es gibt vielfalt davon. Daher kommt die Frage, wie könnten die Geodaten ins Formular eingebaut und visualisert werden.

In diese Arbeit soll anhand eines Beispieles gezeigt werden, wie sich die Geodaten in dem Formularmangementsystem einbauen lassen und welches Nutzen für die Bürger:innen und Behöden mit sich bringt.



 

\section{Vorgehensweise}
%Recherche - Anforderungen - Technologien
Im ersten Abschnitt werden die Grundlagen der Formular- und die Geoinformationssysteme als Konzepte vorgestellt. Als nächstes werden die Geodaten und deren Quellen vorgestellt. Anschließen werden die Entscheidungen getroffen, welche Technologien zur Durchführung der Konzpetentwicklung genutzt wird.

In Hauptteil wird die Ausgangssituation des Praxisbeispiels basierend auf Formularmanagementsystem form.io mit seinen Funktionalitäten beschrieben. 
Darauf wird Konzept erarbeitet, der auf den Anforderungen des Auftraggebers basiert.

In der Auswertung wird die Arbeit evaluiert. Die Herausforderungen und die Vorteile für Auftraggebende betrachtet. Abschließend wird ein Ausblick auf weitere Anwendungsfelder gegeben.
