\chapter{Grundlagen}
\section{Formulare}
Was kann ein Formular (kann daten im Vorfeld überprüfen, validieren)

Welche Daten können ins Verwaltungsprozess eingebunden werden
\section{Formularmanagementsystem}
Was ist Formularmanagementsystem

Thema "Formularmanagementsystem in der öffentlichen Dienstleistungen" - "Das Formularmanagementsystem (FMS) spielt eine wichtige Rolle bei der Digitalisierung öffentlicher Dienstleistungen. Es ermöglicht eine medienbruchfreie digitale Kommunikation zwischen Bürgerinnen, Bürgern, Unternehmen und der Verwaltung. Hier sind einige Informationen zum FMS in Deutschland:

Das ITZBund stellt mit dem FMS eine Formularsoftware für die Bundesverwaltung bereit, die von rund 60 Bundesbehörden für verschiedene Anwendungsbereiche genutzt wird1. Das System unterstützt die Übermittlung einfacher Formulare sowie die Handhabung komplexer Antragsverfahren inklusive des Workflows1.
Das FMS ist eine Komponente zur Umsetzung des Onlinezugangsgesetzes (OZG), das Behörden des Bundes und der Länder verpflichtet, alle zentralen Verwaltungsprozesse bis zum Jahr 2022 in digitaler Form anzubieten1.
Die Bundesfinanzverwaltung bietet über das FMS Online-Dienstleistungen und interaktive Formulare an, die einen vollständigen und medienbruchfreien Datenaustausch ermöglichen2.
Das FMS der Bundesfinanzverwaltung umfasst Formularangebote des Bundesministeriums der Finanzen und seiner zugehörigen Bundesoberbehörden sowie Formulare der Bundeszollverwaltung3.
Diese Informationen zeigen, wie das FMS dazu beiträgt, Verwaltungsprozesse zu vereinfachen und die Effizienz der öffentlichen Verwaltung zu steigern. (bingChat 09.12.2023)"
\section{Geodaten}
\section{Geoinformationssysteme}
Ein Geoinformationssystem (GIS) ist ein Informationssystem, das sich auf Modellierung und digitale Abbildungen von Geoobjekte der realen Welt konzentriert. Diese Systeme erfassen, verarbeiten, speichert, auswerten und visualisieren raumbezogene Informationen. Ein wesentliches Merkmal von GIS ist, dass die verarbeiteten Geoobjekte nicht nur thematische und dynamische Aspekte aufweisen, sondern auch explizit Geometrie und Topologie als implizite Bestandteile beinhalten. Dies erfordert spezielle Werkzeuge und Funktionen für die Verarbeitung raumbezogener Informationen, die von anderen Informationssystemen nicht bereitgestellt werden. \cite[S.~373]{de_lange_geoinformatik_2020} \\
Wie \cite{de_lange_geoinformatik_2020} in seiner vierten Auflage betont, hat sich die Welt der GIS seit den neunziger Jahren grundlegend gewandelt. Er hebt hervor, dass technologische Innovationen, insbesondere seit der Einführung von Smartphones und der zunehmenden Bedeutung des Internets, die Entwicklung der Geoinformatik maßgeblich beeinflusst haben. Diese Fortschritte haben zu neuen Anwendungen wie mobilen GIS und Web-GIS geführt. Es haben sich nicht nur die Technologien, sondern auch die Anwendungsweise der Geoinformatik weiterentwickelt. 